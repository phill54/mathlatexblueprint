\documentclass[11pt,a4paper]{scrartcl}
\usepackage[latin1]{inputenc}
\usepackage{amsmath}
\usepackage{amsfonts}
\usepackage{tikz}
\usepackage{amssymb}
\usepackage{multienum}
\usepackage[ngerman]{babel}

\author{Klasse 10b}
\title{2. Schulaufgabe im Fach Mathematik}
\date {\today}

\begin{document}
\maketitle

\paragraph{Aufgabe}
Der nebenstehende Axialschnitt zeigt den Entwurf eines kleinen Sportpokals. (Drehkörper).\\
Es gilt:\\
\begin{gather}
\overline{FD} = \overline{AD} = 2cm\\
\overline{BN} = 3cm\\
\overline{GN} = 2cm\\
\overline{NJ} = \overline{LK} = 0,8cm\\
\overline{OH} = \overline{EB}\\
\overline{GO} = 1,2cm
\end{gather}

\begin{tikzpicture}[
	cross/.style={
		draw,
		cross out,
		minimum size=2*(#1-1pt),
		inner sep=0pt,
		outer sep=0pt
		},
	x=.5cm,
	y=.5cm
	]
   %Raster zeichnen
   \draw [color=gray!50]  [step=5mm] (-11,-10) grid (11,10);
   
   % Achsen zeichnen
   \draw[->,thick] (-10,0) -- (10,0) node[right] {$x$};
   \draw[->,thick] (0,-9) -- (0,9) node[above] {$y$};
   
   % Achsen beschriften
   \foreach \x in {-1,0,1,2,5}
   \draw (\x,-.1) -- (\x,.1) node[below=4pt] {$\scriptstyle\x$};
   \foreach \y in {-1,0,1,2,5}
   \draw (-.1,\y) -- (.1,\y) node[left=4pt] {$\scriptstyle\y$};
    %Funktionen zeichnen:
    \clip(-11,-9) rectangle (11,9);
    \draw[smooth, domain=-9:9] plot (\x, {2*pow(\x-3,2)-3}) node at (6,8) {$p_1$};  
    \draw[smooth] plot (\x, {-1*pow(\x-3,2)+1}) node[right] {$p_2$};  
    \draw[smooth, domain=-9:9] plot (\x, {-0.4*pow(\x+3,2)+1})
        node at (2.5,-8) {$p_4$};           
\end{tikzpicture} 

\paragraph{Aufgabe 1}
Bestimme die Unbekannte x!

\renewcommand{\regularlisti}{\setcounter{multienumi}{0}%
\renewcommand{\labelenumi}
{\addtocounter{multienumi}{1}\alph{multienumi})}}

\begin{multienumerate}
\mitemxx{$\frac{3-x}{7}=23$}{$24x-43,5=\frac{23}{2-5}$}
\mitemxx{$x^2-2=x^2-7x+3$}{$134-76/3=\frac{34}{4-67x}$}
\end{multienumerate}


\begin{flushright}
Punkte(\_\_\_\_/4)
\end{flushright}

\paragraph{Aufgabe 2}
Bestimme die Unbekannte x!

%\begin{multienumerate}
%\mitemxx{$3x-7=5$}{$4x+5=2$}
%\end{multienumerate}
\begin{align*}
3x-7=5\\
4x+5=2\\
x-4=3-2\\
\end{align*}

\paragraph{Aufgabe 3}
Irgendwann werd ich schon die passende Formierung finden
\begin{align*}
\text{a)} 3x-7&=4 & 34-3x& =7&
b)a_{12}& =b_{12}\\
c)a_{21}& =b_{21}&
d)a_{22}& =b_{22}+c_{22}
\end{align*}

\begin{align*}
x&=y & X&=Y & a&=b+c\\
a) 3x-5&=3 &45x-6x-8&=34x & 4-5x&=4x-3\\
\end{align*}

\end{document}